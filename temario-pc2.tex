% Created 2019-01-14 Mon 12:13
% Intended LaTeX compiler: pdflatex
\documentclass{article}
\usepackage[utf8]{inputenc}
\usepackage[T1]{fontenc}
\usepackage{graphicx}
\usepackage{grffile}
\usepackage{longtable}
\usepackage{wrapfig}
\usepackage{rotating}
\usepackage[normalem]{ulem}
\usepackage{amsmath}
\usepackage{textcomp}
\usepackage{amssymb}
\usepackage{capt-of}
\usepackage{hyperref}
\documentclass[letter,14pt]{article}
\usepackage[letterpaper,right=1.25in,left=1.25in,top=1in,bottom=1in]{geometry}
\usepackage{url}
\usepackage{mathptmx}           % set font type to Times
\usepackage[scaled=.90]{helvet} % set font type to Times (Helvetica for some special characters)
\usepackage{courier}            % set font type to Times (Courier for other special characters)
\author{Profesor: Eric Magar Meurs \small\{\url{emagar@itam.mx}\}}
\date{Lunes y miércoles 16:00--17:30, salón 105}
\title{Política Comparada II\\\medskip
\large Departamento de Ciencia Política ITAM, otoño 2018}
\hypersetup{
 pdfauthor={Profesor: Eric Magar Meurs \small\{\url{emagar@itam.mx}\}},
 pdftitle={Política Comparada II},
 pdfkeywords={},
 pdfsubject={},
 pdfcreator={Emacs 24.5.1 (Org mode 9.1.7)}, 
 pdflang={Spanish}}
\begin{document}

\maketitle
\emph{Aviso}: Es tarea del alumno leer el material (tanto el que aparece en este temario como el que puedo añadir antes de cualquier clase) previo a la fecha indicada.  Mi cátedra da por sentado que lo ha hecho, cubriendo sólo una parte de cada lectura desde una óptica particular. No obstante, en los exámenes evaluaré el conocimiento de las lecturas y la capacidad para discutirlas críticamente. Las fechas de las lecturas pueden cambiar; anunciaré en su momento estos cambios.  

\emph{Objetivo}: Este es el segundo de la serie de tres cursos de política comparada del programa. Está dedicado al estudio de dos temas clásicos en la política comparada de corte institucional: las implicaciones que en la vida de la ciudadanía tienen distintas formas de gobierno democrático; y el análisis de los sistemas electoral y de partidos, haciendo hincapié en su interacción.  

\emph{Horas de oficina}: Lunes y miércoles de 13:00 a 14:00, o con cita. 

\emph{Evaluación}: Habrá un examen parcial y otro final. En su momento anunciaré el formato. Cada uno contará 40\% de la calificación final y el 20\% restante valorará su desempeño en clase, participación y conocimiento de las lecturas.

\emph{Material}: será distribuido en formato electrónico desde la página \url{https://github.com/emagar/pc2}. Si lo consiguen, es recomendable comprar un ejemplar del libro \emph{Modelos de Democracia} (Ed. Ariel) de Arend Lijphart. 

\emph{Días de asueto}: 2019-02-04, 2019-03-18, 2019-04-15 2019-04-17 y 2019-05-01

\emph{Semestre termina}: 2019-05-17

\noindent\rule{\textwidth}{0.5pt}

\section{PARTE I – INTRODUCCION}
\label{sec:org14e3fe9}
\begin{enumerate}
\item Instituciones y el análisis institucional  (16 de enero)
\label{sec:org31f4b78}
\begin{itemize}
\item Myerson, \href{https://github.com/emagar/pc2/blob/master/lecturas/myersonInstAnalysis1995jep.pdf}{"Analysis of democratic Institutions: Structure, Conduct, and Performance,"} 12 pp.
\end{itemize}
\end{enumerate}
\section{PARTE II – FORMAS DE GOBIERNO DEMOCRATICO}
\label{sec:orgbfedb29}
\begin{enumerate}
\item El modelo Westminster de democracia  (21, 23, 28, 30 de enero y 6 de febrero)
\label{sec:org8bf7db2}
\begin{itemize}
\item Lijphart, \href{https://github.com/emagar/ep3/blob/master/lecturas/lijphart-mod-democ}{\emph{Modelos de democracia}}
\begin{itemize}
\item cap. 1 "Introducción," pp. 13-19;
\item cap. 2 "El modelo Westminster de democracia," pp. 21-41 y pp. 117-124.
\end{itemize}
\item Hancock, \emph{Politics in Europe}, 
\begin{itemize}
\item cap. 1.1 "The context of British politics," 19 pp.
\item cap. 1.2 "Where is the power?" 27 pp.
\item cap. 1.3 "Who has the power?" 24 pp.
\item cap. 1.4 "How is the power used?" 15 pp.
\item cap. 1.5 "What is the future of British politics?" 9 pp.
\end{itemize}
\end{itemize}
\item El modelo de democracia consensual  (11 y 13 de febrero)
\label{sec:org1332a14}
\begin{itemize}
\item Lijphart, \emph{Modelos de democracia}, 
\begin{itemize}
\item cap. 3 "El modelo consensual de democracia," pp. 43-57;
\item cap. 4 "Treinta y seis democracias", pp. 59-70.
\end{itemize}
\item The Economist: "A special case: a survey of Switzerland," pp. 3-18.
\end{itemize}
\item Tipos de sistemas constitucionales  (18 de febrero)
\label{sec:orgb3a8ee6}
\begin{itemize}
\item Lijphart, \emph{Modelos de democracia}, cap. 7 "Relaciones entre el ejecutivo y el legislativo: modelos de predominio y de equilibrio de poder," pp. 117-139.
\item Linz, "The Perils of Presidentialism," pp. 51-69.
\end{itemize}
\item Sistemas presidenciales  (20, 25, 27 de febrero, y 4 de marzo)
\label{sec:orgb5210e2}
\begin{itemize}
\item Constitución Política de los Estados Unidos Mexicanos, entera (compre un ejemplar).
\item Hamilton, Madison y Jay, \emph{El Federalista}, números 10, 51 y 78, 24 pp.
\item Weldon, "Las fuentes políticas del presidencialismo en México," 36 pp.
\item McCubbins, "Government on Lay-Away: Federal Spending and Deficits under Divided Control," 40 pp.
\item Magar, "El inmovilismo democrático a revisión," 14 pp.
\item Cheibub, Przeworski y Saiegh, "Government Coalitions and Legislative Success Under Presidentialism and Parliamentarism" 20 pp.
\item Jacobson, \emph{The Politics of Congressional Elections}, caps. 1-3, 50 pp.
\item Shugart y Carey, \emph{Presidents and Assemblies}, cap. 8 "Assessing the powers of the presidency," 19 pp.   (Lectura optativa.)
\end{itemize}
\textbf{EXAMEN PARCIAL: se programa y entrega alrededor del 9 de marzo (anunciaré el formato con anterioridad)}
\item Sistemas Parlamentarios  (6, 11 y 13 de marzo)
\label{sec:org740294a}
\begin{itemize}
\item Lijphart, \emph{Modelos de democracia}, 
\begin{itemize}
\item cap. 6 "Gabinetes: concentración frente a división del poder ejecutivo," 21 pp.
\item cap. 11 "Parlamentos y congresos: concentración frente a división del poder legislativo," 13 pp.
\end{itemize}
\item Hancock, \emph{Politics in Europe}, 
\begin{itemize}
\item cap. 3.1 "The context of German politics," 28 pp.
\item cap. 3.2 "Where is the power?" 18 pp.
\item cap. 3.3 "Who has the power?" 28 pp.
\item cap. 3.4 "How is the power used?" 14 pp.
\item cap. 3.5 "What is the future of German politics?" 15 pp.
\end{itemize}
\end{itemize}
\item Sistemas semi-presidenciales  (20 y 25 de marzo)
\label{sec:orga4d7597}
\begin{itemize}
\item Duverger, "A New Political System Model: Semi-Presidential Government," 7 pp.
\item Schleiter y Morgan-Jones "Review Article: Citizens, Presidents and Assemblies: The Study of Semi-Presidentialism beyond Duverger and Linz," 21 pp.
\item Hancock, \emph{Politics in Europe}, 
\begin{itemize}
\item cap. 2.1 "The context of French politics," 15 pp.
\item cap. 2.2 "Where is the power?" 27 pp.
\item cap. 2.3 "Who has the power?" 42 pp.
\item cap. 2.4 "How is the power used?" 12 pp.
\item cap. 2.5 "What is the future of French politics?" 18 pp.
\end{itemize}
\end{itemize}
\end{enumerate}
\section{PARTE III – PARTIDOS Y ELECCIONES}
\label{sec:orga5b0cdb}
\begin{enumerate}
\item Sistemas electorales  (27 de marzo y 1 de abril)
\label{sec:orgb937479}
\begin{itemize}
\item Lijphart, \emph{Modelos de democracia}, cap. 8 "Sistemas electorales: método de mayoría absoluta y mayoría relativa frente a representación proporcional," 23 pp.
\item Lijphart, \emph{Electoral Systems and Party Systems}, 
\begin{itemize}
\item cap. 1 "Goals and methods," 9 pp.
\item cap. 2 "Electoral systems: types, patterns, trends," 46 pp.
\end{itemize}
\item Broz y Maliniak "Malapportionment, Gasoline Taxes, and the United Nations Framework Convention on Climate Change" 37 pp.
\item Jones, "A Guide to the Electoral Systems of the Americas," 16 pp.
\item Jones, "A Guide to the Electoral Systems of the Americas: An Update," 3 pp.
\end{itemize}
\item El sistema de partidos  (3 y 8 de abril)
\label{sec:org72290c5}
\begin{itemize}
\item Beck, Party \emph{Politics in America}, "Parties and party systems," 32 pp.
\end{itemize}
\item El número de partidos  (10 y 22 de abril)
\label{sec:org9fce3e5}
\begin{itemize}
\item Beck, Party Politics in America, "The American Two-Party System," 32 pp.
\item Lijphart, \emph{Modelos de democracia}, cap. 5 "Sistemas de partidos: modelos bipartidistas y multipartidistas," 23 pp.
\item Mainwaring, "Presidentialism, Multipartism, and Democracy," 30 pp.
\item Molinar, "Counting the Number of Parties: An Alternative Index," 18 pp.
\item Taagepera, "Supplementing the Effective Number of Parties," 7 pp.
\end{itemize}
\item Patrones de competencia partidista  (24 y 29 de abril)
\label{sec:org8bf2993}
\begin{itemize}
\item Magar, Rosenblum y Samuels, "On the absence of centripetal incentives in double-member districts: The case of Chile," 25 pp.
\item Sartori, \emph{Partidos y sistemas de partidos}, 
\begin{itemize}
\item cap. 5 "El criterio numérico," 8 pp.
\item cap 6 "Sistemas competitivos," 92 pp.
\end{itemize}
\item LaPalombara, \emph{Democracy, Italian Style}, pp. 1-8, 16-24, 117-143.
\item Capoccia, "Anti-System Parties: A Conceptual Reassessment," 26 pp.
\end{itemize}
\item Partidos como organizaciones  (6 de mayo)
\label{sec:org2ac800e}
\begin{itemize}
\item Riordon, \emph{Plunkitt of Tamany Hall}, entero, 135 pp.
\item OJO: esta clase será más larga. Nos reuniremos de 15:00 a 17:30 para ver y discutir la película The Last Hurrah de J. Ford (1958) o All the King’s Men de R. Rossen (1949).
\end{itemize}
\item La interacción entre los sistemas electoral y de partidos  (8 y 13 de mayo)
\label{sec:org3f66329}
\begin{itemize}
\item Duverger, "El dualismo de los partidos" y "El multipartidismo," 47 pp.
\end{itemize}
Lijphart, \emph{Electoral Systems and Party Systems}, 
\begin{itemize}
\item cap. 3 "Disproportionality, multipartism, and majority victories," 21 pp.
\item cap. 4 "Changes in elections rules in the same country," 26 pp.
\item cap. 5 "Bivariate and multivariate analyses," 23 pp.
\item App. A "Proportional representation formulas," 7 pp.
\end{itemize}
\begin{itemize}
\item Magar, "Gubernatorial coattails in Mexican congressional elections" 35 pp.
\end{itemize}
\end{enumerate}
\section{PARTE IV – RECAPITULACION}
\label{sec:orgf2e5875}
\begin{enumerate}
\item Instituciones, convivencia política y método comparativo (15 de mayo)
\label{sec:orgab4ea21}
\begin{itemize}
\item Lijphart, \emph{Modelos de democracia}, 
\begin{itemize}
\item cap. 14 "Mapa conceptual bidimensional de la democracia," 23 pp.
\item cap. 16 "Calidad de la democracia y una democracia más benigna y benévola," 21 pp.
\end{itemize}
\end{itemize}
\end{enumerate}
\end{document}
